\documentclass[11pt]{article}
\usepackage{amsmath}
\usepackage{amsthm}
\usepackage{listings}
\usepackage{xcolor}
\usepackage{hyperref}
\usepackage{natbib}

\title{Specification for Visored}
\author{AI Math Autoformalization Team}
\date{\today}

\begin{document}

\maketitle
\tableofcontents

\section{Introduction}
\subsection{Purpose}
This document specifies the syntax and semantics of the Formal Mathematics Programming Language (FMPL), a language designed for automated theorem proving and mathematical formalization.

\section{Related Work}

\paragraph{Naproche} In \cite{Lon2021TheIN}, the authors describe the syntax and semantics of the Naproche proof assistant, which is a natural language proof assistant based on the Isabelle theorem prover.
\section{Goals}
\begin{itemize}
    \item Provide a rigorous foundation for expressing mathematical concepts
    \item Enable automated reasoning and proof verification
    \item Bridge the gap between informal and formal mathematics
    \item Support integration with existing theorem provers
\end{itemize}

\section{Syntax}

The syntax of visored is a subset of LaTeX. In fact, visored has no syntactic stage of its own. It's based on the syntactic output of my personal LaTeX parser.

\subsection{Lexical Elements}
% Basic tokens, keywords, identifiers, etc.

\subsection{Grammar}
% Formal grammar specification using BNF or similar notation

\section{Semantics}
\subsection{Type System}
% Type rules, type checking, type inference

\subsection{Operational Semantics}
% Execution model, evaluation rules

\section{Standard Library}
% Built-in functions and types

\section{Examples and Use Cases}
% Code examples demonstrating language features

\bibliographystyle{plainnat}
\bibliography{references}

\appendix

% \section{Standard Library}
% \subsection{Item Paths}
% \begin{tabular}{|p{0.3\textwidth}|p{0.6\textwidth}|}
%     \hline
%     \textbf{Path} & \textbf{Description}     \\
%     \hline
%     std::path     & Standard path operations \\
%     \hline
% \end{tabular}

% \subsection{Basic Types}
% \begin{tabular}{|p{0.3\textwidth}|p{0.6\textwidth}|}
%     \hline
%     \textbf{Type} & \textbf{Description}         \\
%     \hline
%     bool          & Boolean type (true/false)    \\
%     int           & Integer numbers              \\
%     float         & Floating-point numbers       \\
%     string        & Text sequences               \\
%     list<T>       & Ordered collection of type T \\
%     set<T>        & Unique collection of type T  \\
%     map<K,V>      & Key-value associations       \\
%     option<T>     & Optional value of type T     \\
%     result<T,E>   & Success T or error E value   \\
%     vector<T>     & Mathematical vector          \\
%     matrix<T>     & Mathematical matrix          \\
%     \hline
% \end{tabular}

% \subsection{Basic Functions}
% \begin{tabular}{|p{0.3\textwidth}|p{0.6\textwidth}|}
%     \hline
%     \textbf{Function} & \textbf{Description}     \\
%     \hline
%     add(x, y)         & Addition operation       \\
%     sub(x, y)         & Subtraction operation    \\
%     mul(x, y)         & Multiplication operation \\
%     div(x, y)         & Division operation       \\
%     pow(x, y)         & Exponentiation           \\
%     sqrt(x)           & Square root              \\
%     log(x)            & Natural logarithm        \\
%     exp(x)            & Exponential function     \\
%     abs(x)            & Absolute value           \\
%     floor(x)          & Floor function           \\
%     ceil(x)           & Ceiling function         \\
%     round(x)          & Rounding function        \\
%     min(x, y)         & Minimum of two values    \\
%     max(x, y)         & Maximum of two values    \\
%     \hline
% \end{tabular}

% \subsection{Collection Operations}
% \begin{tabular}{|p{0.3\textwidth}|p{0.6\textwidth}|}
%     \hline
%     \textbf{Function} & \textbf{Description}               \\
%     \hline
%     map(f, xs)        & Apply function to each element     \\
%     filter(p, xs)     & Keep elements satisfying predicate \\
%     reduce(f, xs)     & Reduce collection using function   \\
%     zip(xs, ys)       & Pair corresponding elements        \\
%     flatten(xss)      & Flatten nested collections         \\
%     sort(xs)          & Sort collection                    \\
%     reverse(xs)       & Reverse collection order           \\
%     length(xs)        & Collection size                    \\
%     contains(xs, x)   & Test element membership            \\
%     \hline
% \end{tabular}

% \subsection{String Operations}
% \begin{tabular}{|p{0.3\textwidth}|p{0.6\textwidth}|}
%     \hline
%     \textbf{Function}    & \textbf{Description}      \\
%     \hline
%     concat(s1, s2)       & String concatenation      \\
%     split(s, sep)        & Split string on separator \\
%     trim(s)              & Remove whitespace         \\
%     lowercase(s)         & Convert to lowercase      \\
%     uppercase(s)         & Convert to uppercase      \\
%     replace(s, old, new) & Replace substring         \\
%     \hline
% \end{tabular}

% \subsection{Mathematical Functions}
% \begin{tabular}{|p{0.3\textwidth}|p{0.6\textwidth}|}
%     \hline
%     \textbf{Function} & \textbf{Description} \\
%     \hline
%     sin(x)            & Sine function        \\
%     cos(x)            & Cosine function      \\
%     tan(x)            & Tangent function     \\
%     asin(x)           & Inverse sine         \\
%     acos(x)           & Inverse cosine       \\
%     atan(x)           & Inverse tangent      \\
%     sinh(x)           & Hyperbolic sine      \\
%     cosh(x)           & Hyperbolic cosine    \\
%     tanh(x)           & Hyperbolic tangent   \\
%     \hline
% \end{tabular}

% \subsection{Linear Algebra Operations}
% \begin{tabular}{|p{0.3\textwidth}|p{0.6\textwidth}|}
%     \hline
%     \textbf{Function} & \textbf{Description} \\
%     \hline
%     dot(v1, v2)       & Vector dot product   \\
%     cross(v1, v2)     & Vector cross product \\
%     norm(v)           & Vector norm          \\
%     transpose(m)      & Matrix transpose     \\
%     det(m)            & Matrix determinant   \\
%     inv(m)            & Matrix inverse       \\
%     eigenvals(m)      & Matrix eigenvalues   \\
%     eigenvecs(m)      & Matrix eigenvectors  \\
%     \hline
% \end{tabular}

% \subsection{Resolution Rules}
% \begin{tabular}{|p{0.3\textwidth}|p{0.6\textwidth}|}
%     \hline
%     \textbf{Rule} & \textbf{Description}      \\
%     \hline
%     resolve\_path & Path resolution algorithm \\
%     \hline
% \end{tabular}

% \subsection{Error Handling}
% \begin{tabular}{|p{0.3\textwidth}|p{0.6\textwidth}|}
%     \hline
%     \textbf{Function} & \textbf{Description}      \\
%     \hline
%     ok(value)         & Create success result     \\
%     err(error)        & Create error result       \\
%     is\_ok(result)    & Test if result is success \\
%     is\_err(result)   & Test if result is error   \\
%     unwrap(result)    & Extract success value     \\
%     catch(result, f)  & Handle error case         \\
%     \hline
% \end{tabular}

% \subsection{Dispatch Rules}
% \begin{tabular}{|p{0.3\textwidth}|p{0.6\textwidth}|}
%     \hline
%     \textbf{Rule} & \textbf{Description}        \\
%     \hline
%     dispatch\_fn  & Function dispatch mechanism \\
%     \hline
% \end{tabular}

% \subsection{Type Conversion}
% \begin{tabular}{|p{0.3\textwidth}|p{0.6\textwidth}|}
%     \hline
%     \textbf{Function} & \textbf{Description}    \\
%     \hline
%     to\_int(x)        & Convert to integer      \\
%     to\_float(x)      & Convert to float        \\
%     to\_string(x)     & Convert to string       \\
%     to\_bool(x)       & Convert to boolean      \\
%     parse\_int(s)     & Parse string as integer \\
%     parse\_float(s)   & Parse string as float   \\
%     \hline
% \end{tabular}

% \subsection{IO Operations}
% \begin{tabular}{|p{0.3\textwidth}|p{0.6\textwidth}|}
%     \hline
%     \textbf{Function}          & \textbf{Description}     \\
%     \hline
%     read\_file(path)           & Read file contents       \\
%     write\_file(path, content) & Write to file            \\
%     print(x)                   & Print to standard output \\
%     input()                    & Read from standard input \\
%     \hline
% \end{tabular}


\end{document}
